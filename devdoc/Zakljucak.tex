\chapter{Zaključak i budući rad}
		
	
	Zadatak naše grupe je bio napraviti aplikaciju za šetanje pasa iz udruga. Aplikacija treba omogućiti registraciju udruga i šetača, dodavanje pasa te rezervaciju šetnji.  Projekt smo radili otprilike 16 tjedana te se dinamika u tih 16 tjedana značajno mijenjala.\par
	
	Projekt je bio podijeljen u dvije faze koje su bile određene ocjenjivanjem našeg projekta. U prvoj fazi, dokumentacija je nosila značajno više bodova od same aplikacije, dok je u drugoj fazi bilo obrnuto. Prva faza je započela dobivanjem projektnog zadatka, uvodnim sastankom te zatim upoznavanjem sa pomoćnim alatima i framework-ovima pomoću kojih smo napravili aplikaciju. Također, u prvoj fazi se značajno radilo na slaganju baze podataka te je ona opisana u dokumentaciji. Ipak, kako je značajni dio prve predaje činila dokumentacija, većina fokusa je bila na njoj. Bilo je bitno definirati specifikaciju programske potpore - aktore, funkcionalne i nefunkcionalne zahtjeve. To nam je uvelike pomoglo za izgradnju same aplikacije jer nam je služilo kao plan tj. kostur aplikacije. U svakom trenutku smo znali što i kako trebamo promijeniti jer je sve bilo već smišljeno. Također, opisali smo arhitekturu i framework-ove te dodali sekvencijske dijagrame, obrasce uporabe te dijagrame razreda.\par
	
	U drugoj fazi projekta fokus je bio na aplikaciji. Druga je faza također bila mnogo dinamičnija i intenzivnija faza jer smo gotovo cijelu aplikaciju sagradili zapravo u drugoj, kraćoj fazi. Nitko od članova grupe nije bio otprije upoznat s alatima tako da su svi bili primorani izdvojiti značajno vrijeme kako bi uopće mogli započeti projekt. Zatim, kako je projekt odmicao bilo je sve lakše jer smo se navikli na ponašanje spomenutih alata. Funkcionalnosti je bilo mnogo te je izgradnja nekih tekla očekivanim tokom, dok su neke trajale značajno dulje nego očekivano. Ipak, sve smo zadatke stigli ispuniti. Uz aplikaciju, postoji i značajan dio dokumentacije u drugoj fazi. Dodali smo testove programskog rješenja, opisali korištene tehnologije i alate te napisali upute za puštanje u pogon. Također dodali smo četiri dijagrama: dijagram stanja, dijagram komponenti, dijagram aktivnosti te dijagram razmještaja.\par
	
	Komunikacija među članovima se održava putem Whatsappa te povremeno na Discordu. Na Discordu su bili virtualni sastanci dok je Whatsapp grupa služila za svakodnevno informiranje o napretku i zahtjevima projekta.\par
	
	Sudjelovanje na ovom projektu je bilo vrlo intenzivno i dinamično. Bilo je vrlo naporno, ali i korisno jer smo napravili mnogo u malo vremena. Također, upoznali smo se s novim alatima koje ćemo sigurno ponovno koristiti. Još jedna dobra strana projekta je što smo se sami organizirali te smo mogli vidjeti kakve to prednosti i mane nosi.
		

		
		
		\eject 