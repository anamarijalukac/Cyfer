\chapter{Specifikacija programske potpore}
		
	\section{Funkcionalni zahtjevi}
			
			\textbf{\textit{dio 1. revizije}}\\
			
			\textit{Navesti \textbf{dionike} koji imaju \textbf{interes u ovom sustavu} ili  \textbf{su nositelji odgovornosti}. To su prije svega korisnici, ali i administratori sustava, naručitelji, razvojni tim.}\\
				
			\textit{Navesti \textbf{aktore} koji izravno \textbf{koriste} ili \textbf{komuniciraju sa sustavom}. Oni mogu imati inicijatorsku ulogu, tj. započinju određene procese u sustavu ili samo sudioničku ulogu, tj. obavljaju određeni posao. Za svakog aktora navesti funkcionalne zahtjeve koji se na njega odnose.}\\
			
			
			\noindent \textbf{Dionici:}
			
			\begin{packed_enum}
				
				\item Voditelji udruga
				\item Šetači pasa (registrirani korisnici)			
				\item Zaposlenici i volonteri u udrugama
				\item Administrator
				\item Razvojni tim
				
			\end{packed_enum}
			\vspace{5mm}
			
			\noindent \textbf{Aktori i njihovi funkcionalni zahtjevi:}
			
			
			\begin{packed_enum}
				\item  \underbar{Javni posjetitelj (inicijator) može:}
				
				\begin{packed_enum}
					
					\item pregledati listu udruga na naslovnoj stranici
					\item odabrati udrugu te pregledati: 
					\begin{packed_enum}
						
						\item  detalje profila udruge:
						\begin{packed_enum}
							\item ime udruge
							\item voditelj udruge
							\item kontakt: email adresa i broj mobitela
							\item lokacija
							\item OIB udruge
							\item IBAN udruge - u slučaju da netko želi napraviti donaciju
						\end{packed_enum}
						\item  listu pasa iz te udruge koji su raspoloživi za šetnju
					\end{packed_enum}
				
					\item odabrati profil psa iz liste pasa te pregledati detalje profila psa: 
					\begin{packed_enum}
						\item ime psa
						\item vrsta psa (ako je poznata)
						\item slika psa
						\item opis psa (osobnost, izgled)
						\item dob psa
						\item raspored odnosno raspoloživost psa za određeni vremenski period (datum i vrijeme) 
						\item vrsta šetnje za koju je pas predodređen (skupna ili individualna)
					\end{packed_enum} 
					\item otvoriti statistiku svih pasa raspoloživih za šetnju i vidjeti koji pas se najmanje šteao, odnosno kojem psu je šetnja najpotrebnija
					\item registrirati se u sustav kao građanin - za stvaranje korisničkog računa potrebni su mu:
					\begin{packed_enum}
						\item ime i prezime
						\item e-mail adresa
						\item lozinka 
					\end{packed_enum}
					\item registrirati u sustav svoju udrugu - za stvaranje korisničkog računa potrebni su mu:
					\begin{packed_enum}
						\item ime i prezime
						\item e-mail adresa
						\item lozinka 
						\item naziv udruge
						\item OIB udruge
					\end{packed_enum}
					\item  otvoriti rang-listu svih registriranih šetača poredanu s obzirom na broj šetnji, broj pasa te duljinu šetnje koju su odradili u proteklih mjesec dana
					
				\end{packed_enum}
				\vspace{5mm}
			
				\item  \underbar{Prijavljeni građanin (inicijator)} preuzima sve funkcionalnosti javnog posjetitelja te može dodatno:
				\begin{packed_enum}
					\item prijaviti se u sutav (s e-mailom i lozinkom)
					\item uređivati vlastiti profil
					\item obrisati vlastiti profil
					\item odabrati psa te na njegovom profilu prijaviti se za šetnju
					\item pregledati vlastiti raspored šetnji te skinuti (eng. download) raspored za odabrani dan, tjedan ili mjesec, u PDF obliku
					\item pregledati vlastitu statistiku šetanja
					\item označiti vlastite statistike šetanja kao \underbar{javne} kako bi podaci građana dospjeli na rang listu na javnoj stranici
				\end{packed_enum}
				\vspace{5mm}
			
				\item  \underbar{Prijavljena udruga (inicijator)} preuzima sve funkcionalnosti javnog posjetitelja te može dodatno:
				\begin{packed_enum}
					\item prijaviti se u sutav (s e-mailom i lozinkom)
					\item uređivati vlastiti profil
					\item dodavati i brisati pse iz liste raspoloživih pasa te udruge
					\item uređivati profile pasa koji su iz te udruge 
					\item obrisati vlasitti profil
				\end{packed_enum}
				\vspace{5mm}
			
				\item  \underbar{Administrator(inicijator)} može:
				\begin{packed_enum}
					\item vidjeti popis svih registriranih korisnika i udruga njihovih osobnih podataka
					\item dodati ili obrisati udruge
					\item ???
				\end{packed_enum}
				\vspace{5mm}
			
				\item  \underbar{Baza podataka (sudionik)}:
				\begin{packed_enum}
					\item pohranjuje sve podatke o korisnicima i udrugama 
					\item ???
				\end{packed_enum}
			
			\end{packed_enum}
			
			\eject 
			
			
				
			\subsection{Obrasci uporabe}
				
				\textbf{\textit{dio 1. revizije}}
				
				\subsubsection{Opis obrazaca uporabe}
					\textit{Funkcionalne zahtjeve razraditi u obliku obrazaca uporabe. Svaki obrazac je potrebno razraditi prema donjem predlošku. Ukoliko u nekom koraku može doći do odstupanja, potrebno je to odstupanje opisati i po mogućnosti ponuditi rješenje kojim bi se tijek obrasca vratio na osnovni tijek.}\\
					

					\noindent \underbar{\textbf{UC$<$broj obrasca$>$ -$<$ime obrasca$>$}}
					\begin{packed_item}
	
						\item \textbf{Glavni sudionik: }$<$sudionik$>$
						\item  \textbf{Cilj:} $<$cilj$>$
						\item  \textbf{Sudionici:} $<$sudionici$>$
						\item  \textbf{Preduvjet:} $<$preduvjet$>$
						\item  \textbf{Opis osnovnog tijeka:}
						
						\item[] \begin{packed_enum}
	
							\item $<$opis korak jedan$>$
							\item $<$opis korak dva$>$
							\item $<$opis korak tri$>$
							\item $<$opis korak četiri$>$
							\item $<$opis korak pet$>$
						\end{packed_enum}
						
						\item  \textbf{Opis mogućih odstupanja:}
						
						\item[] \begin{packed_item}
	
							\item[2.a] $<$opis mogućeg scenarija odstupanja u koraku 2$>$
							\item[] \begin{packed_enum}
								
								\item $<$opis rješenja mogućeg scenarija korak 1$>$
								\item $<$opis rješenja mogućeg scenarija korak 2$>$
								
							\end{packed_enum}
							\item[2.b] $<$opis mogućeg scenarija odstupanja u koraku 2$>$
							\item[3.a] $<$opis mogućeg scenarija odstupanja  u koraku 3$>$
							
						\end{packed_item}
					\end{packed_item}
				
				\noindent \underbar{\textbf{UC1 - Pregled profila udruga i pasa}}
				\begin{packed_item}
					
					\item \textbf{Glavni sudionik:} javni posjetitelj
					\item  \textbf{Cilj:} Otvoriti profil pojedine udruge ili psa
					\item  \textbf{Sudionici:} Baza podataka
					\item  \textbf{Preduvjet:} -
					\item  \textbf{Opis osnovnog tijeka:}
					
					\item[] \begin{packed_enum}
						
						\item Korisnik sa naslovne strane odabire udrugu koju želi proučiti
						\item Iz liste pasa te udruge korisnik odabire psa koji ga zanima
						\item Korisnik može proučavati podatke i raspored željenog psa
					\end{packed_enum}
					
					\item  \textbf{Opis mogućih odstupanja:}
					
					\item[] \begin{packed_item}
						
						\item [2.a] Odabir već zauzete email adrese, unos podataka u neispravnom formatu
						\item[] \begin{packed_enum}
							
							\item Sustav obavjestava korisnika o neuspjelom upisu i vraća ga na stranicu za registraciju
							\item Korisnik mijenja potrebne podatke te zavrsava unos ili odustaje od registracije
							
						\end{packed_enum}
					\end{packed_item}
				\end{packed_item}
			
			
	
		
				\noindent \underbar{\textbf{UC3 - Registracija građanina ili udruge u sustav}}
				\begin{packed_item}
					
					\item \textbf{Glavni sudionik:} javni posjetitelj
					\item  \textbf{Cilj:} Stvoriti korisnički račun za pristup sustavu
					\item  \textbf{Sudionici:} Baza podataka
					\item  \textbf{Preduvjet:} -
					\item  \textbf{Opis osnovnog tijeka:}
					
					\item[] \begin{packed_enum}
						
						\item Korisnik odabire opciju (gumb) za registraciju
						\item Korisnik unosi potrebne korisničke podatke (ime, prezime, email adresa i lozinka za građanina te dodatno ime udruge i OIB udruge za udrugu)
						\item Korisnik prima obavijest o uspješnoj registraciji
					\end{packed_enum}
					
					\item  \textbf{Opis mogućih odstupanja:}
					
					\item[] \begin{packed_item}
						
						\item [2.a] Odabir već zauzete email adrese, unos podataka u neispravnom formatu
						\item[] \begin{packed_enum}
							
							\item Sustav obavjestava korisnika o neuspjelom upisu i vraća ga na stranicu za registraciju
							\item Korisnik mijenja potrebne podatke te zavrsava unos ili odustaje od registracije

						\end{packed_enum}
					\end{packed_item}
				\end{packed_item}
			
				\noindent \underbar{\textbf{UC4 - Prijava građanina ili udruge u sustav}}
				\begin{packed_item}
					
					\item \textbf{Glavni sudionik:} registrirani građanin/registrirana udruga
					\item  \textbf{Cilj:} Dobiti pristup odgovarajućem korisničkom sučelju
					\item  \textbf{Sudionici:} Baza podataka
					\item  \textbf{Preduvjet:} Registracija građanina ili udruge u sustav
					\item  \textbf{Opis osnovnog tijeka:}
					
					\item[] \begin{packed_enum}
						
						\item Unos email adrese i lozinke
						\item Potvrda o ispravnosti unesenih podataka
						\item Pristup odgovarajućim korisničkim funkcijama (ovisi prijavljuje li se građanin ili udruga)
					\end{packed_enum}
					
					\item  \textbf{Opis mogućih odstupanja:}
					
					\item[] \begin{packed_item}
						
						\item [2.a] Neispravno ime/lozinka
						\item[] \begin{packed_enum}
							
							\item Sustav obavještava korisnika o neuspjelom upisu i vraća ga na stranicu za prijavu
							
						\end{packed_enum}
					\end{packed_item}
				\end{packed_item}
				
		
				                                                                 
				
					
				\subsubsection{Dijagrami obrazaca uporabe}
					
					\textit{Prikazati odnos aktora i obrazaca uporabe odgovarajućim UML dijagramom. Nije nužno nacrtati sve na jednom dijagramu. Modelirati po razinama apstrakcije i skupovima srodnih funkcionalnosti.}
				\eject		
				
			\subsection{Sekvencijski dijagrami}
				
				\textbf{\textit{dio 1. revizije}}\\
				
				\textit{Nacrtati sekvencijske dijagrame koji modeliraju najvažnije dijelove sustava (max. 4 dijagrama). Ukoliko postoji nedoumica oko odabira, razjasniti s asistentom. Uz svaki dijagram napisati detaljni opis dijagrama.}
				\eject
	
		\section{Ostali zahtjevi}
		
			\textbf{\textit{dio 1. revizije}}\\
		 
			 \textit{Nefunkcionalni zahtjevi i zahtjevi domene primjene dopunjuju funkcionalne zahtjeve. Oni opisuju \textbf{kako se sustav treba ponašati} i koja \textbf{ograničenja} treba poštivati (performanse, korisničko iskustvo, pouzdanost, standardi kvalitete, sigurnost...). Primjeri takvih zahtjeva u Vašem projektu mogu biti: podržani jezici korisničkog sučelja, vrijeme odziva, najveći mogući podržani broj korisnika, podržane web/mobilne platforme, razina zaštite (protokoli komunikacije, kriptiranje...)... Svaki takav zahtjev potrebno je navesti u jednoj ili dvije rečenice.}
			 
			 
			 
	