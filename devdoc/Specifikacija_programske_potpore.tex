\chapter{Specifikacija programske potpore}
	
\section{Funkcionalni zahtjevi}
		
		\textbf{\textit{dio 1. revizije}}\\
		
		\textit{Navesti \textbf{dionike} koji imaju \textbf{interes u ovom sustavu} ili  \textbf{su nositelji odgovornosti}. To su prije svega korisnici, ali i administratori sustava, naručitelji, razvojni tim.}\\
			
		\textit{Navesti \textbf{aktore} koji izravno \textbf{koriste} ili \textbf{komuniciraju sa sustavom}. Oni mogu imati inicijatorsku ulogu, tj. započinju određene procese u sustavu ili samo sudioničku ulogu, tj. obavljaju određeni posao. Za svakog aktora navesti funkcionalne zahtjeve koji se na njega odnose.}\\
		
		
		\noindent \textbf{Dionici:}
		
		\begin{packed_enum}
			
			\item Voditelji udruga
			\item Šetači pasa (registrirani korisnici)			
			\item Zaposlenici i volonteri u udrugama
			\item Administrator
			\item Razvojni tim
			
		\end{packed_enum}
		\vspace{5mm}
		
		\noindent \textbf{Aktori i njihovi funkcionalni zahtjevi:}
		
		
		\begin{packed_enum}
			\item  \underbar{Javni posjetitelj (inicijator) može:}
			
			\begin{packed_enum}
				
				\item pregledati listu udruga na naslovnoj stranici
				\item odabrati udrugu te pregledati: 
				\begin{packed_enum}
					
					\item  detalje profila udruge:
					\begin{packed_enum}
						\item ime udruge
						\item voditelj udruge
						\item kontakt: email adresa i broj mobitela
						\item lokacija
						\item OIB udruge
						\item IBAN udruge - u slučaju da netko želi napraviti donaciju
					\end{packed_enum}
					\item  listu pasa iz te udruge koji su raspoloživi za šetnju
				\end{packed_enum}
			
				\item odabrati profil psa iz liste pasa te pregledati detalje profila psa: 
				\begin{packed_enum}
					\item ime psa
					\item vrsta psa (ako je poznata)
					\item slika psa
					\item opis psa (osobnost, izgled)
					\item dob psa
					\item raspored odnosno raspoloživost psa za određeni vremenski period (datum i vrijeme) 
					\item vrsta šetnje za koju je pas predodređen (skupna ili individualna)
				\end{packed_enum} 
				\item otvoriti statistiku svih pasa raspoloživih za šetnju i vidjeti koji pas se najmanje šteao, odnosno kojem psu je šetnja najpotrebnija
				\item registrirati se u sustav kao građanin - za stvaranje korisničkog računa potrebni su mu:
				\begin{packed_enum}
					\item ime i prezime
					\item e-mail adresa
					\item lozinka 
				\end{packed_enum}
				\item registrirati u sustav svoju udrugu - za stvaranje korisničkog računa potrebni su mu:
				\begin{packed_enum}
					\item ime i prezime
					\item e-mail adresa
					\item lozinka 
					\item naziv udruge
					\item OIB udruge
				\end{packed_enum}
				\item  otvoriti rang-listu svih registriranih šetača poredanu s obzirom na broj šetnji, broj pasa te duljinu šetnje koju su odradili u proteklih mjesec dana
				
			\end{packed_enum}
			\vspace{5mm}
		
			\item  \underbar{Prijavljeni građanin (inicijator)} preuzima sve funkcionalnosti javnog posjetitelja te može dodatno:
			\begin{packed_enum}
				\item prijaviti se u sutav (s e-mailom i lozinkom)
				\item uređivati vlastiti profil
				\item obrisati vlastiti profil
				\item odabrati psa te na njegovom profilu prijaviti se za šetnju
				\item pregledati vlastiti raspored šetnji te skinuti (eng. download) raspored za odabrani dan, tjedan ili mjesec, u PDF obliku
				\item pregledati vlastitu statistiku šetanja
				\item označiti vlastite statistike šetanja kao \underbar{javne} kako bi podaci građana dospjeli na rang listu na javnoj stranici
			\end{packed_enum}
			\vspace{5mm}
		
			\item  \underbar{Prijavljena udruga (inicijator)} preuzima sve funkcionalnosti javnog posjetitelja te može dodatno:
			\begin{packed_enum}
				\item prijaviti se u sutav (s e-mailom i lozinkom)
				\item uređivati vlastiti profil
				\item dodavati i brisati pse iz liste raspoloživih pasa te udruge
				\item uređivati profile pasa koji su iz te udruge 
				\item obrisati vlasitti profil
			\end{packed_enum}
			\vspace{5mm}
		
			\item  \underbar{Administrator(inicijator)} može:
			\begin{packed_enum}
				\item vidjeti popis svih registriranih korisnika i udruga njihovih osobnih podataka
				\item dodati ili obrisati udruge
				\item ???
			\end{packed_enum}
			\vspace{5mm}
		
			\item  \underbar{Baza podataka (sudionik)}:
			\begin{packed_enum}
				\item pohranjuje sve podatke o korisnicima i udrugama 
				\item ???
			\end{packed_enum}
		
		\end{packed_enum}
		
		\eject 
		
		
			
		\subsection{Obrasci uporabe}
		\vspace{5mm}
			
		
			
			\noindent \underbar{\textbf{\hypertarget{UC1}{UC1} - Pregled profila udruga i pasa iz te udruge}}
			\begin{packed_item}
				
				\item \textbf{Glavni sudionik:} Javni posjetitelj
				\item  \textbf{Cilj:} Otvoriti profil pojedine udruge ili psa iz te udruge
				\item  \textbf{Sudionici:} Baza podataka
				\item  \textbf{Preduvjet:} -
				\item  \textbf{Opis osnovnog tijeka:}
				
				\item[] \begin{packed_enum}
					
					\item Korisnik sa naslovne strane odabire udrugu koju želi proučiti
					\item Iz liste pasa te udruge korisnik odabire psa koji ga zanima
					\item Korisnik može proučavati podatke, raspored i statistiku šetanja željenog psa
				\end{packed_enum}
			\end{packed_item}
		
		
			\noindent \underbar{\textbf{\hypertarget{UC2}{UC2} - Pregled liste profila svih pasa}}
			\begin{packed_item}
				
				\item \textbf{Glavni sudionik:} Javni posjetitelj
				\item  \textbf{Cilj:} Otvoriti profil psa iz bilo koje udruge
				\item  \textbf{Sudionici:} Baza podataka
				\item  \textbf{Preduvjet:} -
				\item  \textbf{Opis osnovnog tijeka:}
				
				\item[] \begin{packed_enum}
					
					\item Korisnik iz izborne trake odabire "Lista svih pasa"
					\item Iz liste pasa korisnik odabire psa koji ga zanima
					\item Korisnik može proučavati podatke, raspored i statistiku šetanja željenog psa
				\end{packed_enum}
			\end{packed_item}
		
		
		
			\noindent \underbar{\textbf{UC3 - Pregled statistike svih pasa}}
			\begin{packed_item}
				
				\item \textbf{Glavni sudionik:} Javni posjetitelj
				\item  \textbf{Cilj:} Otvoriti statistiku šetanja svih pasa
				\item  \textbf{Sudionici:} Baza podataka
				\item  \textbf{Preduvjet:} -
				\item  \textbf{Opis osnovnog tijeka:}
				
				\item[] \begin{packed_enum}
					
					\item Korisnik iz izborne trake odabire "Lista svih pasa"
					\item Korisnik odabire opciju "Statistika svih pasa"
					\item Korisnik može proučiti statistiku šetanja svih pasa 
				\end{packed_enum}
			\end{packed_item}
		
		\newpage
			
				
			\noindent \underbar{\textbf{UC4 - Pregled rang-liste svih šetača}}
			\begin{packed_item}
				
				\item \textbf{Glavni sudionik:} Javni posjetitelj
				\item  \textbf{Cilj:} Dobiti uvid u rang-listu svih šetača
				\item  \textbf{Sudionici:} Baza podataka
				\item  \textbf{Preduvjet:} -
				\item  \textbf{Opis osnovnog tijeka:}
				
				\item[] \begin{packed_enum}
					
					\item Korisnik iz izborne trake odabire "Rang-lista šetača"
					\item Aplikacija prikazuje poredak šetača s obzirom na broj šetnji, broj pasa te duljinu šetnji koje su odradili u prethodnih mjesec dana
					\item Korisnik može proučiti rang-listu svih šetača
				\end{packed_enum}
				\item  \textbf{Opis mogućih odstupanja:}
				\item[] \begin{packed_item}
					\item [2.a] U slučaju da nijedan šetač nije označio statistiku svoje šetnje kao javnu, rang lista će biti prazna
				\end{packed_item}
			\end{packed_item}
	
	
	

	
			\noindent \underbar{\textbf{UC5 - Registracija građanina ili udruge u sustav}}
			\begin{packed_item}
				
				\item \textbf{Glavni sudionik:} Javni posjetitelj
				\item  \textbf{Cilj:} Stvoriti korisnički račun za pristup sustavu
				\item  \textbf{Sudionici:} Baza podataka
				\item  \textbf{Preduvjet:} -
				\item  \textbf{Opis osnovnog tijeka:}
				
				\item[] \begin{packed_enum}
					
					\item Korisnik odabire opciju (gumb) za registraciju 
					\item Korisnik bira "Registriraj se kao građanin" ili "Registriraj se kao udruga"
					\item Korisnik unosi potrebne korisničke podatke (ime, prezime, email adresa i lozinka za građanina te dodatno ime udruge i OIB udruge za udrugu)
					\item Korisnik odabire opciju "Stvori korisnički račun"
					\item Korisnik prima obavijest o uspješnoj registraciji
				\end{packed_enum}
				
				\item  \textbf{Opis mogućih odstupanja:}
				
				\item[] \begin{packed_item}
					
					\item [2.a] Unos podataka u neispravnom formatu 
					\item[] \begin{packed_enum}
						
						\item Sustav obavještava korisnika o neispravnim podatcima i vraća ga na stranicu za registraciju
						\item Korisnik mijenja potrebne podatke te završava unos ili odustaje od registracije

					\end{packed_enum}
					\item [2.b] Odabrana email adresa je već zauzeta
					\item[] \begin{packed_enum}
					
						\item Sustav obavještava korisnika o zauzetoj email adresi i vraća ga na stranicu za registraciju
						\item Korisnik mijenja potrebne podatke te završava unos ili odustaje od registracije
					
				\end{packed_enum}
				\end{packed_item}
			\end{packed_item}
		
		
		
			\noindent \underbar{\textbf{UC6 - Prijava građanina ili udruge u sustav}}
			\begin{packed_item}
				
				\item \textbf{Glavni sudionik:} Registrirani građanin/registrirana udruga
				\item  \textbf{Cilj:} Dobiti pristup odgovarajućem korisničkom sučelju
				\item  \textbf{Sudionici:} Baza podataka
				\item  \textbf{Preduvjet:} Registracija građanina ili udruge u sustav
				\item  \textbf{Opis osnovnog tijeka:}
				
				\item[] \begin{packed_enum}
					
					\item Unos email adrese i lozinke
					\item Potvrda o ispravnosti unesenih podataka
					\item Pristup odgovarajućim korisničkim funkcijama (ovisi prijavljuje li se građanin ili udruga)
				\end{packed_enum}
				
				\item  \textbf{Opis mogućih odstupanja:}
				
				\item[] \begin{packed_item}
					
					\item [2.a] Neispravno ime/lozinka
					\item[] \begin{packed_enum}
						
						\item Sustav obavještava korisnika o neuspjelom upisu i vraća ga na stranicu za prijavu
						
					\end{packed_enum}
				\end{packed_item}
			\end{packed_item}
		
		
				\noindent \underbar{\textbf{UC7 - Pregled osobnih podataka korisnika (udruge ili građanina)}}
			\begin{packed_item}
				
				\item \textbf{Glavni sudionik:} Registrirani građanin/registrirana udruga
				\item  \textbf{Cilj:} Pregledati osobne podatke/podatke udruge
				\item  \textbf{Sudionici:} Baza podataka
				\item  \textbf{Preduvjet:} Registrirani korisnik (udruga ili građanin) je prijavljen u sustav
				\item  \textbf{Opis osnovnog tijeka:}
				
				\item[] \begin{packed_enum}	
					\item Korisnik iz izborne trake odabire "Osobni podatci"
					\item Otvara se stranica sa korisnikovim osobnim podatcima
				\end{packed_enum}
			\end{packed_item}
		
		
		
				\noindent \underbar{\textbf{UC8 - Promjena osobnih podataka korisnika}}
			\begin{packed_item}
				
				\item \textbf{Glavni sudionik:} Registrirani građanin/registrirana udruga
				\item  \textbf{Cilj:} Promijeniti osobne podatke/podatke udruge
				\item  \textbf{Sudionici:} Baza podataka
				\item  \textbf{Preduvjet:} Registrirani korisnik (udruga ili građanin) je prijavljen u sustav
				\item  \textbf{Opis osnovnog tijeka:}
				
				\item[] \begin{packed_enum}
					\item Korisnik iz izborne trake odabire "Osobni podatci"
					\item Otvara se stranica sa korisnikovim osobnim podatcima
					\item Korisnik bira "Uredi profil"
					\item Korisnik mijenja podatke
					\item Korisnik bira opciju "Pohrani promjene"
					\item Baza se ažurira
				\end{packed_enum}
				
				\item  \textbf{Opis mogućih odstupanja:}
				
				\item[] \begin{packed_item}
					
					\item [4.a]  Korisnik promijeni svoje osobne podatke, ali ne odabere opciju ”Pohrani
					promjene”
					\item[] \begin{packed_enum}
						
						\item Sustav upozorava korisnika da nije spremio podatke prije izlaska iz prozora
						\item Korisnik se vraća i pohranjuje promjene ili izlazi iz prozora ostavljajući podatke bez promjene
					\end{packed_enum}
				
					\item [4.b]  Korisnik promijeni svoje osobne podatke, ali u neispravan format
					\item[] \begin{packed_enum}
						
						\item Sustav upozorava korisnika da je podatak u neispravnom formatu i ne dopušta pohranu takvih podataka
						\item Korisnik mijenja podatak i pohranjuje promjene ili izlazi iz prozora ostavljajući podatke bez promjene
					\end{packed_enum}
					
						\item [6.b]  Korisnik promijeni svoje osobne podatke, ali je nova email adresa već zauzeta
					\item[] \begin{packed_enum}
						
						\item Sustav upozorava korisnika da je email adresa već zauzeta i da promjene ne mogu biti pohranjene
						\item Korisnik mijenja email adresu i pohranjuje promjene ili izlazi iz prozora ostavljajući podatke bez promjene
					\end{packed_enum}
				\end{packed_item}
			\end{packed_item}
		
		
				\noindent \underbar{\textbf{UC9 - Brisanje korisničkog računa (udruge ili građanina)}}
			\begin{packed_item}
				
				\item \textbf{Glavni sudionik:} Registrirani građanin/registrirana udruga
				\item  \textbf{Cilj:} Izbrisati vlastiti korisnički račun
				\item  \textbf{Sudionici:} Baza podataka
				\item  \textbf{Preduvjet:} Registrirani korisnik (udruga ili građanin) je prijavljen u sustav
				\item  \textbf{Opis osnovnog tijeka:}
				
				\item[] \begin{packed_enum}	
					\item Korisnik iz izborne trake odabire "Osobni podatci"
					\item Otvara se stranica sa korisnikovim osobnim podatcima
					\item Korisnik bira "Uredi profil"
					\item Korisnik bira "Izbriši korisniči račun"
					\item Sustav upozorava korisnika da je brisanje korisničkog računa trajno te provjerava je li siguran
					\item Korisnik potvrđuje brisanje
					\item Korisnički račun se briše iz baze podataka
					\item Otvara se naslovna stranica
				\end{packed_enum}
				\item  \textbf{Opis mogućih odstupanja:}
				
				\item[] \begin{packed_item}
					
					\item [5.a]  Korisnik poništi brisanje računa
					\item[] \begin{packed_enum}
						\item Račun se ne briše te se baza ne ažurira
						\item Korisnik je ostavljen na stranici "Uredi profil"
					\end{packed_enum}
				\end{packed_item}
			\end{packed_item}
		
			
				\noindent \underbar{\textbf{UC10 - Dodavanje novog profila psa u listu prijavljene udruge}}
			\begin{packed_item}
				
				\item \textbf{Glavni sudionik:} Registrirana udruga
				\item  \textbf{Cilj:} Dodati profil nekog psa iz te (prijavljene) udruge
				\item  \textbf{Sudionici:} Baza podataka
				\item  \textbf{Preduvjet:} Registrirana udruga je prijavljena u sustav
				\item  \textbf{Opis osnovnog tijeka:}
				
				\item[] \begin{packed_enum}
					\item Korisnik iz izborne trake odabire "Moji psi"
					\item Otvara se lista svih pasa iz te udruge
					\item Korisnik bira opciju "Dodaj novog psa"
					\item Otvara se stranica za upis podataka o novom psu 
					\item Korisnik upiše podatke, opcionalno dodaje i sliku
					\item Korisnik bira opciju "Dodaj psa"
					\item Baza se ažurira
				\end{packed_enum}
				
				\item  \textbf{Opis mogućih odstupanja:}
				
				\item[] \begin{packed_item}
					
					\item [4.a]  Korisnik upiše podatke o psu, ali ne odabere opciju "Dodaj psa"
					\item[] \begin{packed_enum}
						
						\item Sustav upozorava korisnika da nije spremio nove podatke prije izlaska iz prozora
						\item Korisnik se vraća i pohranjuje podatke ili izlazi iz prozora bez dodavanja novog psa
					\end{packed_enum}
					
					\item [4.b]  Korisnik upiše podatke o psu, ali novi podatci su u neispravanom formatu
					\item[] \begin{packed_enum}
						
						\item Sustav upozorava korisnika da je podatak (ili više njih) u neispravnom formatu i ne dopušta pohranu takvih podataka
						\item Korisnik mijenja podatak/e i pohranjuje ih ili izlazi iz prozora bez dodavanja novog psa
					\end{packed_enum}
				\end{packed_item}
			\end{packed_item}
		
		
				\noindent \underbar{\textbf{UC11 - Uređivanje profila psa neke udruge}}
			\begin{packed_item}
				
				\item \textbf{Glavni sudionik:} Registrirana udruga
				\item  \textbf{Cilj:} Urediti profil nekog psa iz te (prijavljene) udruge
				\item  \textbf{Sudionici:} Baza podataka
				\item  \textbf{Preduvjet:} Registrirana udruga je prijavljena u sustav
				\item  \textbf{Opis osnovnog tijeka:}
				
				\item[] \begin{packed_enum}
					\item Korisnik iz izborne trake odabire "Moji psi"
					\item Otvara se lista svih pasa iz te udruge
					\item Korisnik bira psa čiji profil želi urediti
					\item Korisnik mijenja podatke 
					\item Korisnik bira opciju "Pohrani promjene"
					\item Baza se ažurira
				\end{packed_enum}
				
				\item  \textbf{Opis mogućih odstupanja:}
				
				\item[] \begin{packed_item}
					
					\item [4.a]  Korisnik promijeni podatke o psu, ali ne odabere opciju ”Pohrani
					promjene”
					\item[] \begin{packed_enum}
						
						\item Sustav upozorava korisnika da nije spremio podatke prije izlaska iz prozora
						\item Korisnik se vraća i pohranjuje promjene ili izlazi iz prozora ostavljajući podatke bez promjene
					\end{packed_enum}
					
					\item [4.b]  Korisnik promijeni podatke o psu, ali novi podatci su u neispravanom formatu
					\item[] \begin{packed_enum}
						
						\item Sustav upozorava korisnika da je podatak (ili više njih) u neispravnom formatu i ne dopušta pohranu takvih podataka
						\item Korisnik mijenja podatak/e i pohranjuje promjene ili izlazi iz prozora ostavljajući podatke bez promjene
					\end{packed_enum}
				\end{packed_item}
			\end{packed_item}
		
		
			\noindent \underbar{\textbf{UC12 - Brisanje profila psa iz liste neke udruge}}
		\begin{packed_item}
			
			\item \textbf{Glavni sudionik:} Registrirana udruga
			\item  \textbf{Cilj:} Obrisati profil nekog psa iz liste pasa te (prijavljene) udruge
			\item  \textbf{Sudionici:} Baza podataka
			\item  \textbf{Preduvjet:} Registrirana udruga je prijavljena u sustav
			\item  \textbf{Opis osnovnog tijeka:}
			
			\item[] \begin{packed_enum}
				\item Korisnik iz izborne trake odabire "Moji psi"
				\item Otvara se lista svih pasa iz te udruge
				\item Korisnik bira psa čiji profil želi obrisati
				\item Korisnik bira opciju "Obriši profil psa" 
				\item Sustav šalje upit korisniku je li siguran
				\item Korisnik potvrđuje brisanje
				\item Profil tog psa se briše iz baze podataka
				\item Otvara se stranica "Moji psi"
			\end{packed_enum}
			\item  \textbf{Opis mogućih odstupanja:}
			
			\item[] \begin{packed_item}
				
				\item [5.a]  Korisnik poništi brisanje profila psa
				\item[] \begin{packed_enum}
					\item Profil psa se ne briše te se baza ne ažurira
					\item Korisnik je ostavljen na stranici "Uredi profil psa"
				\end{packed_enum}
			\end{packed_item}
		\end{packed_item}
	
	
	
		\noindent \underbar{\textbf{UC13 - Rezervacija termina šetnje}}
	\begin{packed_item}
		
		\item \textbf{Glavni sudionik:} Registrirani građanin (šetač)
		\item  \textbf{Cilj:} Šetač rezervira psa i termin u kojem će obaviti šetnju tog psa
		\item  \textbf{Sudionici:} Baza podataka
		\item  \textbf{Preduvjet:} Registrirani građanin (šetač) je prijavljen u sustav
		\item  \textbf{Opis osnovnog tijeka:}
		
		\item[] \begin{packed_enum}
			\item Šetač dolazi na stranicu za odabir profila psa preko:
				\item[] \begin{packed_enum}
					\item profila neke udruge (\hyperlink{UC1}{UC1})
					\item liste svih pasa (\hyperlink{UC2}{UC2})
				\end{packed_enum}
			\item Šetač odabire psa kojeg želi šetati
			\item Šetač dobiva uvid u slobodne termine odabranog psa te bira neki
			\item Sustav šalje potvrdu rezervacije termina šetaču
			\item Nakon uspješne rezervacije, termin za odabranog psa vidljiv je na kalendaru šetaču
		\end{packed_enum}
		\item  \textbf{Opis mogućih odstupanja:}
		
		\item[] \begin{packed_item}
			
			\item [4.a]  Odabrani pas nema slobodnih termina za šetnje
			\item[] \begin{packed_enum}
				\item Šetač izlazi iz profila psa i može tražiti novog psa za šetnju
			\end{packed_enum}
		\end{packed_item}
	\end{packed_item}


	\noindent \underbar{\textbf{UC14 - Pregled vlastitih statistika šetnji}}
	\begin{packed_item}
		
		\item \textbf{Glavni sudionik:} Registrirani građanin (šetač)
		\item  \textbf{Cilj:} Pregledati svoju statistiku šetnji u određenom razdoblju
		\item  \textbf{Sudionici:} Baza podataka
		\item  \textbf{Preduvjet:} Registrirani građanin (šetač) je prijavljen u sustav
		\item  \textbf{Opis osnovnog tijeka:}
		
		\item[] \begin{packed_enum}
			\item Šetač bira opciju "Moj profil" na izbornoj traci odnosno odlazi na stranicu svog profila
			\item Šetač odabire opciju "Moja statistika" na stranici profila
			\item Šetač bira razdoblje u kojem želi pregledati statistiku
			\item Statistika je prikazana te ju šetač može proučiti 
		\end{packed_enum}
	\end{packed_item}

\newpage
	
	\noindent \underbar{\textbf{UC15 - Označavanje statistike šetnji kao javne}}
	\begin{packed_item}
		
		\item \textbf{Glavni sudionik:} Registrirani građanin (šetač)
		\item  \textbf{Cilj:} Učiniti svoju statistiku šetnji javno dostupnom kako bi se mogla prikazati na rang-listi šetača
		\item  \textbf{Sudionici:} Baza podataka
		\item  \textbf{Preduvjet:} Registrirani građanin (šetač) je prijavljen u sustav
		\item  \textbf{Opis osnovnog tijeka:}
		
		\item[] \begin{packed_enum}
			\item Šetač bira opciju "Moj profil" na izbornoj traci odnosno odlazi na stranicu svog profila
			\item Šetač odabire opciju "Moja statistika" na stranici profila te se prikazuje stranica njegove statistike
			\item Šetač odabire opciju "Označi svoju statistiku javnom" 
			\item Njegova statistika je prikazana na javnoj rang-listi šetača
		\end{packed_enum}
	\end{packed_item}




	\noindent \underbar{\textbf{UC16 - Pregled i skidanje (eng.download) kalendara registriranog građanina}}
	\begin{packed_item}
		
		\item \textbf{Glavni sudionik:} Registrirani građanin (šetač)
		\item  \textbf{Cilj:} Pregledati svoj kalendar odnosno raspored rezerviranih šetnji i skinuti raspored za dan, mjesec ili godinu u PDF-u
		\item  \textbf{Sudionici:} Baza podataka
		\item  \textbf{Preduvjet:} Registrirani građanin (šetač) je prijavljen u sustav
		\item  \textbf{Opis osnovnog tijeka:}
		
		\item[] \begin{packed_enum}
			\item Šetač bira opciju "Moj profil" na izbornoj traci odnosno odlazi na stranicu svog profila
			\item Šetač odabire opciju "Moj kalendar" na stranici profila
			\item Šetač može birati želi li vidjeti raspored za dan, mjesec ili godinu
			\item Šetač može skinuti odabrani raspored u PDF-u
		\end{packed_enum}
	\end{packed_item}
		
		
		
		
		
		
		
		
			
			
	
                                                                 
			
				
			\subsubsection{Dijagrami obrazaca uporabe}
				
				\textit{Prikazati odnos aktora i obrazaca uporabe odgovarajućim UML dijagramom. Nije nužno nacrtati sve na jednom dijagramu. Modelirati po razinama apstrakcije i skupovima srodnih funkcionalnosti.}
			\eject		
			
		\subsection{Sekvencijski dijagrami}
			
			\textbf{\textit{dio 1. revizije}}\\
			
			\textit{Nacrtati sekvencijske dijagrame koji modeliraju najvažnije dijelove sustava (max. 4 dijagrama). Ukoliko postoji nedoumica oko odabira, razjasniti s asistentom. Uz svaki dijagram napisati detaljni opis dijagrama.}
			\eject

	\section{Ostali zahtjevi}
	
		\textbf{\textit{dio 1. revizije}}\\
 
	 \textit{Nefunkcionalni zahtjevi i zahtjevi domene primjene dopunjuju funkcionalne zahtjeve. Oni opisuju \textbf{kako se sustav treba ponašati} i koja \textbf{ograničenja} treba poštivati (performanse, korisničko iskustvo, pouzdanost, standardi kvalitete, sigurnost...). Primjeri takvih zahtjeva u Vašem projektu mogu biti: podržani jezici korisničkog sučelja, vrijeme odziva, najveći mogući podržani broj korisnika, podržane web/mobilne platforme, razina zaštite (protokoli komunikacije, kriptiranje...)... Svaki takav zahtjev potrebno je navesti u jednoj ili dvije rečenice.}
	 
	 
	 
